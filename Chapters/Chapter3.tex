% Indicate the main file. Must go at the beginning of the file.
% !TEX root = ../main.tex

%----------------------------------------------------------------------------------------
% CHAPTER TEMPLATE
%----------------------------------------------------------------------------------------

\chapter{Methodik: Historische Diskursanalyse mit korpuslinguistischen Methoden} % Main chapter title

\label{Chapter3} % Change X to a consecutive number; for referencing this chapter elsewhere, use \ref{ChapterX}

%----------------------------------------------------------------------------------------
% SECTION 1
%----------------------------------------------------------------------------------------

\section{Korpuslinguistik}

Die Korpuslinguistik versteht sich als empirische Disziplin der Linguistik, die Sprache anhand einer grösseren Menge an beobachteten und gesammelten Sprachdaten untersucht. In Saussurschen Termini ausgedrückt fokussiert sie auf \textit{langue}, Sprachgebrauch, statt \textit{parole}, das Sprachsystem (\cite[30-31]{lemnitzer_korpuslinguistik_2015}).

Die Annahme, dass sich Sprache nicht anhand theoretischer Überlegungen, sondern primär anhand real beobachteten Sprachgebrauchs analysieren lässt, macht sie zu einer empirisch vorgehenden Disziplin (\cite[21]{lemnitzer_korpuslinguistik_2015}). 

\subsection{Was ist ein Korpus?}
Um den Sprachgebrauch analysieren zu können, müssen grosse Mengen von Sprachdaten gesammelt werden. Eine solche Zusammenstellung grösserer Mengen an Text wird Korpus genannt. Korpora werden in mehreren Teildisziplinen der Linguistik sowie auch anderen Wissenschaften, die mit Textdaten arbeiten, verwendet. Linguistische Korpora unterscheiden sich in der Regel dadurch, dass sie versuchen, Sprachgebrauch (sei es allgemeinsprachlich oder die Sprache spezieller Gruppen) repräsentativ zu dokumentieren und dafür grosse Mengen möglichst unterschiedlicher authentischer Sprachdaten zu sammeln. Häufig werden Korpora annotiert, sodass den sprachlichen Rohdaten Wortarten, syntaktische Funktionen, Lemmatisierungen oder andere Anmerkungen hinzugefügt wurden. Da reine Textdaten meist mehrdeutig sind, aber ihr Kontext im Satz häufig eine eindeutige Bestimmung erlaubt, können an einen annotierten Korpus abstraktere Anfragen gestellt werden (\cite[22-23, 56]{stefanowitsch_anatol_corpus_2020}, 
\cite[58-59]{lemnitzer_korpuslinguistik_2015}). Eine wichtige Vorbedingung für die Annotierung eines Korpus ist seine Zerlegung in Worteinheiten. Dies wird als Tokenisierung bezeichnet und nicht nur Wörter, sondern auch Zahlen, Satzzeichen und andere Symbole annotiert. Dafür wird häufig davon ausgegangen, dass eine von Leerzeichen umgebene Zeichenfolge eine Einheit darstellt, die als \textit{Token} bezeichnet wird. Nicht jedes Token ist also ein Wort im herkömmlichen Verständnis (\cite[61-62]{lemnitzer_korpuslinguistik_2015}).

Erste Korpora wurden bereits vor dem Aufkommen elektronischer Datenverarbeitungssysteme erstellt. Aufgrund des hohen Aufwands einer manuellen Auswertung grösserer Textmengen blieben sie auf wenige Gebiete beschränkt, wie etwa die Konkordanzen religiöser Texte. Erste elektronische Korpora entstanden seit den 1960er-Jahren (\cite[52]{tognini-bonelli_corpus_2001}).

\subsection{«Corpus-based» und «corpus-driven»}
Aufgrund des hohen Aufwands, der für eine manuelle Erstellung und Auswertung eines Korpus nötig wäre, wurden die ersten Korpora zunächst meist als Belegsammlungen und zur Bestätigung von aus der Theorie abgeleiteten Hypothesen eingesetzt. Dieser deduktive Ansatz wird heute auch als «corpus-based» oder «korpusgestützt» bezeichnet. 

Die wachsende Leistungsfähigkeit und leichtere Verfügbarkeit von Computern machte aber seit den 1990er-Jahren die maschinelle Auswertung elektronischer Korpora einfacher. In der Folge wurde ein induktives Vorgehen bei der Auswertung der Korpora möglich. Dieser «corpus-driven» oder «korpusbasierte» Ansatz versucht nicht, aus der Theorie formulierte Vermutungen mit Korpusdaten zu bestätigen, sondern umgekehrt, die Analysekategorien, Theorien und verallgemeinerbare Aussagen aus den Korpusdaten selbst abzuleiten (\cite[65]{tognini-bonelli_corpus_2001}).

Als Teilgebiet der Linguistik sind die Methoden der Korpuslinguistik zunächst entwickelt worden, um Fragen zu beantworten, die sich auf Sprache beziehen. Kritisch für das hier verfolgte Interesse ist, dass korpuslinguistische Analysen sprachliche Phänomene rein auf statistischer Grundlage untersuchen. Was sie liefern sind Zahlen und Aggregate. Fundstellen und Belege sind aus ihrem sprachlichen und para-sprachlichen Kontext entrissen und werden statistisch untersucht, nicht aufgrund inhaltlicher oder semantischer Zusammenhänge (\cite[61-62]{kupietz_korpuslinguistik_2018}). Um nachvollziehen zu können, wie Korpora auch Fragen zu sozialem Handeln, Kultur und Gesellschaft beantworten können, also auch die semantische Dimension von Sprache erfassen können, müssen wir uns dem Konzept des Diskurses zuwenden.

\section{Diskurs- und Korpuslinguistik}

\subsection{Was ist ein Diskurs?}

Der Begriff Diskurs wird je nach theoretischem Zugang und Erkenntnisinteresse unterschiedlich definiert. Als einflussreich hat sich Michel Foucaults Verständnis von Diskursen erwiesen. Für Foucault sind Diskurse charakterisiert als regulierte Systeme, die bestimmte Aussagen über ein Thema überhaupt möglich machen. Diskurse sind Netze von aufeinander bezogenen Aussagen und Handlungen. Damit sind sie wesentlich für die Produktion gesellschaftlich geteilter Wahrheiten und überhaupt Erkenntnismöglichkeiten. Gesellschaftliche Prozesse, regelrechte Sagbarkeitsregime, regulieren, wer innerhalb eines Diskurses etwas sagen und was gesagt werden kann. Sie sind zentral für die Produktion einer gesellschaftlich geteilten Wirklichkeit, die durch die Diskurse immer sprachlich vermittelt wird. Es gibt keine Möglichkeit, eine «Wahrheit» hinter den Diskursen zu erfassen, denn jedes Verständnis der Welt, das wir uns machen, wird bereits durch Sinnproduktionen und Diskurse geprägt (\cite{foucault_ordnung_2021}).

Foucaults Ansatz der Untersuchung von Diskursen zielte darauf zu untersuchen, welche Instanzen die Regelmässigkeit des Diskurses sicherstellten, indem sie Zugang und Ausschluss zu Diskursen regelten. Sein Ansatz ist anschlussfähig an wissenssoziologische Konzepte, die der Sprache eine zentrale Rolle bei der «sozialen Konstruktion der Wirklichkeit» zuwiesen. Da in Gemeinschaften, die über die \textit{face to face}-Community hinausgehen, Wirklichkeit immer sozial konstruiert und zuvorderst sprachlich vermittelt wird (\cite{berger_gesellschaftliche_2018}), ist die sprachliche Wirklichkeit auch die einzig zugängliche Wirklichkeit.

Die erkenntnistheoretischen Folgen eines solchen Verständnisses sind, dass der Sprache eine zentrale Rolle als Analyseobjekt zukommt, um die Wirklichkeit zu beschreiben. Sprechakte und Sprechweisen erfolgen regelmässig. Eine Analyse der innerhalb eines Diskurses vorkommenden sprachlichen Muster könnte daher Aufschluss darüber geben, welche Regeln einen Diskurs beherrschten und was damit den Diskursteilnehmer*innen als sag- und denkbar und damit überhaupt als «wahr» vorkommt (\cite[32]{bubenhofer_sprachgebrauchsmuster_2009}).

\subsection{Korpuslinguistische Untersuchung von Diskursen}

Foucault entwickelte keine explizite Methode einer Diskursanalyse, seine Ideen wurden aber in vielen Disziplinen rezipiert. So haben sowohl die Linguistik als auch die Geschichtswissenschaft Methoden einer Diskursanalyse entwickelt. Da Diskurse als zusammenhängende Netze sprachlicher Äusserungen verstanden werden, also als eine grosse Menge an sprachlichen Äusserungen, bietet sich eine korpuslinguistische Analyse von Diskursen an. Wenn wir davon ausgehen, dass die in einem Korpus versammelten Sprachdaten einen diskursiven Zusammenhang aufweisen und Diskurse «wirklichkeitsstiftend» sind, dann erlaubt eine Untersuchung dieses Korpus Rückschlüsse auf die Prozesse, Möglichkeiten und Grenzen der sozialen Konstruktion der Wirklichkeit. Wir gehen also davon aus, dass die sprachliche Oberfläche, die wir in einem Diskurs vorfinden, nicht zufällig so entstanden ist, sondern dass ihre Produktion einer Regelmässigkeit unterliegt.

\begin{displayquote}
    Signifikant häufig auftretende sprachliche Muster können deshalb als das Ergebnis rekurrenter Sprachhandlungen der Sprecherinnen und Sprecher gedeutet werden, in die typische Verwendungskontexte, Handlungsziele und Interpretationsrahmen eingeschrieben sind (\cite[63]{kupietz_korpuslinguistik_2018}).
\end{displayquote}

\subsection{Konzepte der Diskursanalyse in der Geschichtswissenschaft}

Die Geschichtsschreibung ist eine der paradigmatischen Geisteswissenschaften. Ihre Verwissenschaftlichung im 19.~Jahrhundert hängt eng mit der Entwicklung der historisch-kritischen Methode der Quelleninterpretation zusammen. Die Hermeneutik prägt das Fach stark und im Fokus steht oft immer noch das Verstehen und die umfassende Rekonstruktion des Einzelfalls ohne das Ziel, daraus induktive Schlüsse zu ziehen (\cite{dilthey_aufbau_1992}).

Quantitative Ansätze haben in der Geschichtswissenschaft daher unterschiedliche Konjunkturen erfahren. Besonders die Sozialgeschichte, die im deutschsprachigen Raum vor allem mit der sogenannten Bielefelder Schule verbunden wurde, betonte gegenüber der traditionellen Historiografie die Bedeutung von sozialen und wirtschaftlichen Strukturen gegenüber individuellem Handeln. Gesellschaftliche Makroprozesse sollten nicht zuletzt durch quantitative Analysen erklärt werden. Auch international erlebte ab etwa 1960 eine deutlich quantitativ ausgerichtete Wirtschafts- und Sozialgeschichte (\textit{Cliometrics}) eine starke Verbreitung. Der \textit{cultural turn}, der in den 1980er-Jahren einsetzte, forderte die Sozialgeschichte jedoch heraus. Das Interesse der «neuen Kulturgeschichte» an Alltags- und Mikrogeschichte, historischer Anthropologie und ihre konstruktivistische Erkenntnistheorie, die nach den Sinngebungsprozessen und Deutungen historischer Akteure fragte und sich methodologisch eher an der Ethnologie als Leitwissenschaft orientierte, wurde als Gegensatz zur Sozialgeschichte gedeutet, die sich als Teil der Sozialwissenschaften verstand (\cite{nathaus_sozialgeschichte_2012}).

Seitdem nehmen quantitative Ansätze eine eher periphere Rolle in der Geschichtswissenschaft ein, wenn man von der Wirtschaftsgeschichte absieht. Die starke Tradition der Hermeneutik im Fach, die mit der historisch-kritischen Methode das verstehende Lesen und sorgfältige Rekontextualisierung von Einzelquellen als Kernmethode des Fachs verteidige und die Dominanz der Kulturgeschichte mit einer konstruktivistischen Epistemologie seit den 1990er-Jahren liessen und lassen viele Historiker*innen mit Skepsis auf quantitative Ansätze blicken (\cite[S.~A.1-21]{hohls_digital_2018}).

\subsection{Begriffsgeschichte und Historische Semantik}

Entsprechend der Bedeutung schriftlicher Quellen für die Geschichtswissenschaft, gibt es eine lange Tradition an Ansätzen, die Schnittmengen mit linguistischen Fragestellungen haben. Im deutschsprachigen Raum ist zunächst an die Begriffsgeschichte und die historische Semantik zu denken, die anhand von Schlüsselbegriffen historischen Wandel und Bedeutungsverschiebungen von Begriffen untersuchte (\cite{koselleck_begriffsgeschichte_1972, koselleck_geschichtliche_1972}). Mit ihrem Interesse an einzelnen Wörtern in ihren konkreten Verwendungszusammenhängen und dem Versuch, den semantischen Gehalt des Begriffs in den jeweiligen Belegstellen zu identifizieren und daraus wiederum auf sozial- und kulturhistorischen Wandel, etwa Mentalitätswandel oder sozialstrukturelle Umbrüche und Kontinuitäten zu schliessen, ist die Begriffsgeschichte sehr nahe an dem hier verfolgten linguistischen Vorhaben. Angesichts der Nähe des begriffshistorischen Zugriffs zur Korpus- und Diskurslinguistik erstaunt, dass der interdisziplinäre Austausch zwischen den beiden Disziplinen nur zögerlich in Gang kam (\cite{busse_historische_2000}).

Ein wichtiger Unterschied liegt jedoch in der Aufmerksamkeit für die sprachliche Oberfläche. Die Begriffsgeschichte ist meist nicht an Wörtern interessiert, sondern an Konzepten und Institutionen, die mit den Begriffen ausgedrückt werden. Aus linguistischer Sicht ist die Auswahl und das Verständnis der Untersuchungsgegenstände der Begriffsgeschichte problematisch. Dietrich Busse hat die linguistische Kritik an Programm und Umsetzung der Begriffsgeschichte in Gestalt der «Geschichtlichen Grundbegriffe» pointiert zusammengefasst. Ihr Begriffsbegriff bleibt unbestimmt und kann sich nicht von der impliziten Annahme lösen, dass die Begriffe doch auf dahinterliegende Dinge verweisen, die das eigentliche Forschungsobjekt bilden. Das führt zu einer theoretisch nur schwer begründbaren Unterscheidung zwischen untersuchungsrelevanten Begriffen und einem grossen Rest an blossen Wörtern und Äusserungen. Die reale begriffsgeschichtliche Praxis präsentiert sich weniger als semantische Analyse, sondern meist als herkömmliche Ideengeschichte, die sich auf normative Quellen zur Erklärung des jeweiligen Begriffs stützt (\cite[50-71]{busse_historische_1987}). 

Wichtig für diese Arbeit ist Busses Hinweis, dass eine historische Semantik die Verwendungskontexte miteinbeziehen muss, in der die Begriffe historisch ihren Sinn entfaltet haben. Er verweist dazu auf die Diskurskonzeption Michel Foucaults, die sich im gleichen Zeitraum auch in den Geschichtswissenschaften etablierte.

\subsection{Historische Diskursanalyse}

Das Werk Michel Foucaults wurde in den 1970ern, angesichts der sozialhistorischen Dominanz, noch zögerlich rezipiert. Seit den 1990er-Jahren ist sein Werk aber von Historiker*innen auf sehr verschiedene Weisen als theoretische Orientierung verwendet worden. Foucaults Diskursarchäologie hat dabei viele Arbeiten angeregt, die versuchten, soziale Wirklichkeit als historische Diskurse zu rekonstruieren, indem die wechselseitige Konstitution von Diskursen und Diskursteilnehmer*innen in den Blick genommen wurde (\cite{landwehr_historische_2018}).

Während es der Begriffsgeschichte darum geht, den semantischen Gehalt der Begriffe zu verstehen, vor allem auch in seinem Wandel über die Zeit hinweg, zielt die Diskursanalyse im Gefolge Foucaults weniger auf den Sinn eines Wortes, sondern auf die Position einer Aussage in einem Diskurssystem: Welche diskursiven Formationen machten es überhaupt möglich, eine Aussage so zu treffen und sie als sinnvoll verstanden zu wissen? In welchem Zusammenhang mit dem «Vorher» und «Nachher» des Diskurses steht dieser Ausdruck (\cite[159-160]{sarasin_sozialgeschichte_2012})?

\subsection{Digital Humanities und Interesse für quantitative Methoden}

Die seit den 2000er-Jahren stetig wachsenden Digital Humanities, computergestützten Geisteswissenschaften, hat auch die Geschichtswissenschaft erfasst. Mit zunehmender Leistungsfähigkeit der Computer und der Entwicklung benutzerfreundlicher Software wird auch in der hermeneutisch geprägten Geschichtswissenschaft die maschinelle Auswertung von Quellen wieder vermehrt diskutiert. Neben den klassischen seriellen Quellen der Sozialgeschichte werden auch Analysemethoden zur Auswertung grosser Textmengen rezipiert (z.~B. «Topic Modelling», \cite{graham_getting_2012}). Gerade niederschwellige Tools wie Googles \textit{n-gram}-Viewer wecken bei Historiker*innen die Faszination, wie alte Fragestellungen neu bearbeitet werden können. Auch hier liegt das Interesse aber mehr in der Anwendung als in einem Verständnis der linguistischen Hintergründe (zumindest kommt die Diskussion bei \cite{sarasin_sozialgeschichte_2012} ohne einen Verweis auf die linguistische Diskussion aus). Erste Versuche, neuere korpuslinguistische Methoden für die Begriffsgeschichte fruchtbar zu machen, wurden indes mittlerweile unternommen (\cite{schwandt_digitale_2018, kamper_diskurslinguistik_2018}). Dass schon die aus korpuslinguistischer Sicht einfache Methoden von Historiker*innen als anregend innovativ empfunden werden und neue Fragen und Einsichten ermöglichen, zeigen Burckhardt u.~a., die ausgewählte Begriffe in der DDR-Presse diachron verfolgen (\cite{burckhardt_distant_2019}).

%----------------------------------------------------------------------------------------
% SECTION 3
%----------------------------------------------------------------------------------------

\section{Korpuslinguistik für kulturanalytische Fragestellungen}

Die Voraussetzungen für eine korpuslinguistisch inspirierte Untersuchung historischer Diskurse sind also gegeben. Im Folgenden werden kurz die wichtigsten theoretischen Grundannahmen skizziert. Ausgegangen wird davon, dass gesellschaftliche Wirklichkeit sprachlich konstituiert ist. Jede Handlung kann nur durch sprachlich geleitete Konstruktion von Sinn ein \textit{soziales} Handeln werden. Dadurch, dass diese Sprachhandlungen aufeinander bezogen sind, entsteht ein Diskurs, der die geteilten Konstruktionen von Sinn ermöglicht. Jede sprachliche Äusserung ist eine Handlung im Diskurs. Sie schliesst an vorhergehende Äusserungen und Sinnkonstruktionen an und beeinflusst folgende Sinnkonstruktionen und damit zukünftiges Handeln. Die Analyse der sprachlichen Äusserungen innerhalb eines Diskurses erlaubt daher, von der sprachlichen Oberfläche auf die soziale Wirklichkeit zu schliessen (\cite[63]{lemke_text_2015}).

\subsection{Modellierung des Diskurses in einem Korpus}

Entscheidend für die Aussagekraft einer korpuslinguistischen Untersuchung ist die Zusammenstellung des Korpus. Linguistische Korpora versuchen eine grösstmögliche Repräsentativität herzustellen, das heisst den Sprachgebrauch umfassend abzubilden. In der Praxis ist eine absolute Repräsentativität nicht erreichbar, weshalb in der Regel eine möglichst grosse Diversität an Sprachdaten und Quellen im Korpus angestrebt wird (\cite[34-35]{stefanowitsch_anatol_corpus_2020}). Bei begriffsgeschichtlichen und diskursanalytischen Korpora stellte sich noch dringender die Frage nach der Repräsentativität der Korpusauswahl. Angesichts der begrenzten Anzahl an Texten, die mit einem qualitativen Ansatz untersucht werden können, stellt sich die Frage, ob die jeweils analysierten Texte genügend typisch sind, um Aussagen über die Struktur des Diskurses machen zu können (\cite[101-103]{landwehr_historische_2018}).

Allgemeinsprachliche Korpora bieten sich auch für die Untersuchung von Spezialthemen an. Wenn ein Diskurs tatsächlich wirkmächtig ist, sollten seine Sprechmuster und Sagbarkeitsregimes auch in einem allgemeinen Sprachgebrauch Niederschlag finden. Für Fragen an Spezial- oder Gegendiskurse kann es aber auch erkenntnisfördernd sein, einen möglichst spezifischen Korpus zu kompilieren. So lassen sich etwa einfacher Erkenntnisse gewinnen, die im Rauschen der anderen Diskurse untergehen würden (\cite{hodel_kleine_2013}).

Die Frage der Repräsentativität stellt sich bei zu kulturanalytischen Zwecken zusammengestellten Korpora in verschärftem Masse. Die Auswahl der Texte des Korpus, die als zu einem Diskurs gehörig verstandenen werden, unterliegt der Fragestellung und damit eben der Annahme, dass sie in einem historisch-semantischen Zusammenhang stehen. Das ist unvermeidbar, aber es unterwirft die Texte einem Vorverständnis (\cite[15-16]{busse_ist_1994}). Eine gänzlich «naiv»-induktive Diskursanalyse, bei der die Daten «für sich selber sprechen» ist daher theoretisch unmöglich. Diskurse, ob gegenwärtige oder historische, existieren nicht unabhängig für sich als beobachtbare Objekte, sondern sie entstehen, weil Forscherinnen annehmen, dass eine nach gewissen Kriterien bestimmbare Menge an Texten in einem diskursiven Zusammenhang stehen. Jedes Korpus, das für eine Diskursanalyse modelliert wird, ist abhängig von der Fragestellung, dem Erkenntnisinteresse und den zur Verfügung stehenden Daten und Auswertungsmöglichkeiten (\cite[143-144]{dreesen_diskurslinguistik_2019}).

\subsection{Operationalisierung: Frequenzanalyse und Kollokationen}\label{Kollokationen}

Als hauptsächliche Analysemethoden werden in dieser Arbeit Frequenzanalysen und Kollokationsprofile angewendet. Frequenzanalysen sind eine recht einfache Auswertungsmöglichkeit. Sie zählen, wie häufig ein Token innerhalb des Korpus auftaucht. Das Verfahren ist unter Historiker*innen seit Googles \textit{n-Gram Viewer} vertraut. Dieser ermöglicht eine Frequenzanalyse von Begriffen in den von Google Books gescannten Büchern ermöglicht (\cite{sarasin_sozialgeschichte_2012}). Die Häufigkeiten können absolut oder relativ per 1 Million Wörter ausgegeben werden. Kategorisiert man die Treffer nach Erscheinungsjahr des Textes ergeben sich schnell Einblicke in die Konjunkturen des Suchbegriffs: Wird er im Laufe der Zeit mehr oder weniger verwendet? Gibt es auffällige Peaks? In diesem Fall ist es meist sinnvoller, die relative Häufigkeit der Token zu betrachten, um über die Jahre hinweg ungleich verteilte Textmengen auszugleichen.

Werden bei Frequenzanalysen einzelne Token gezählt, so wird bei der Kollokationsanalyse das Umfeld eines Suchbegriffs in den Blick genommen. Bubenhofer summiert das verbreitete «emipirische Verständnis» von Kollokationen als «Paare von Worteinheiten […], die innerhalb einer bestimmten Distanz zueinander kookkurrieren und eine statistisch feststellbare Bindung zueinander aufweisen.» Das Mass für die Bindung wird meist statistisch bemessen, nämlich danach, ob gewisse Wortpaare häufiger in einem Korpus auftauchen, als aufgrund einer zufälligen Verteilung statistisch erwartbar ist. Die Teile einer Kollokation müssen nicht unbedingt in direkter Nachbarschaft stehen, sondern die Suche kann auch auf Token im Umfeld ausgedehnt werden. Theoretisch offen ist die Definition auch für Einheiten aus mehr als zwei Wörtern, n-Gramme, deren Assoziation aber statistisch schwieriger zu konzeptualisieren ist (\cite[69-70]{bubenhofer_kollokationen_2017}). 

Dem Einsatz von Kollokationen in der kulturanalytischen Korpuslinguistik liegt die Annahme der linguistischen Pragmatik zugrunde, dass der semantische Gehalt von Wörtern von ihren Verwendungskontexten bestimmt wird (\cite[44-46]{bubenhofer_sprachgebrauchsmuster_2009}). Sie sind die sprachlichen Muster, in denen sich soziale Phänomene ausdrücken oder diese (mit-)konstituieren. Vom überzufällig häufigen gemeinsamen Auftritt von Wörtern kann auf Bedeutungszuschreibungen und Kontexte geschlossen werden (\cite[63]{kupietz_korpuslinguistik_2018}).