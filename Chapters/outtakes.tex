\begin{figure}
    \begin{filecontents*}{./Data/anzahl_schueler.csv}
        \end{filecontents*}   
    \begin{tikzpicture}
        \begin{axis}[
            width=\linewidth,
            xlabel={Jahr},
            ylabel={Treffer per Mio. Wörter},
            xmin=1800, xmax=1870,
            ymin=0, ymax=300,
            ymajorgrids=true,
            x tick label style={rotate=35, anchor=east, yshift= -5pt},
            legend pos=north west,
            height = 10cm,
            xtick pos = left,
            ytick pos = left,
        ]
    %    \addplot [smooth, thick, zhawgray, dashed] table[x=Jahr, y=Treffer, col sep = comma]{./Data/durchschnittlich.csv};
    %    \addlegendentry{Treffer}
        \addplot [smooth, thick, zhawblue] table[x=Jahr, y=Treffer per Mio. Wörter, col sep = comma]{./Data/anzahl_schueler.csv};
    %    \addlegendentry{Treffer per Mio. Wörter}
        \end{axis}
        \end{tikzpicture}
        \caption{Relative Häufigkeit für \texttt{[pos= "CARD"][lemma= ".*schüler\-|.*schülerin|\-Knabe|\-Mädchen|\-Tochter|\-Kind|\-Schüler|\-Schülerin|Bube"]}}
        \label{figure:5-3}
    \end{figure}
    
    \begin{figure}
        \begin{filecontents*}{./Data/durchschnittlich.csv}
        \end{filecontents*}   
    \begin{tikzpicture}
        \begin{axis}[
            width=\linewidth,
            xlabel={Jahr},
            ylabel={Treffer per Mio. Wörter},
            ymin=0, ymax=250,
            ymajorgrids=true,
            xmin=1800, xmax=1870,
            x tick label style={font=\normalsize, rotate=90, anchor=east, yshift= -5pt},
            legend pos=north west,
            height = 7cm,
            xtick pos = left,
            ytick pos = left,
        ]
    %    \addplot [smooth, thick, zhawgray, dashed] table[x=Jahr, y=Treffer, col sep = comma]{./Data/durchschnittlich.csv};
    %    \addlegendentry{Treffer}
        \addplot [mark=o,only marks, mark options={scale=1.5}] table[x=Jahr, y=Treffer per Mio. Wörter, col sep = comma]{./Data/durchschnittlich.csv};
    %    \addlegendentry{Treffer per Mio. Wörter}
        \end{axis}
        \end{tikzpicture}
    
    \end{figure}

    \subsection{Lehrerinnen und Lehrer}

    Ebenso wie die Schülerinnen und Schüler wurden auch die Lehrerinnen und Lehrer gezählt (Abbildung~\ref{figure:5-4}). Im Vergleich zu Suchbegriffen für Zahlen in Verbindung mit \textit{Schüler} etc. (Abbildung~\ref{figure:5-1}) fällt auf, dass die relativen Frequenzen niedriger sind und eine Zahl gefolgt von \textit{Lehrerin} oder \textit{Lehrer}. Um 1835 bis 1840 sehen wir ein etwas höheres Niveau an Treffern, das danach sinkt und ab 1855 sich wieder erhöht, in einzelnen Jahren sogar deutlich. 
    
    \begin{figure}
        \begin{tikzpicture}    
        \pgfplotstableread[col sep=comma]{./Data/lehrerinnen.csv}{\loadedtable}
        \begin{axis}[
            ybar,
            width=\linewidth,
            ymajorgrids=true,
            xlabel=Jahr,
            ylabel=Treffer per Mio.~Wörter,
            legend style={at={(xticklabel cs:.5)},anchor=north},% Legende abhänging von xticks positioniert
            x tick label style={rotate=90,anchor=east},
            xmin = 1799,
            xmax = 1871,
            height = 7cm,
            bar width=3pt,
            xtick pos = left,
            ytick pos = left,
            ymin = 0,
          ]
        \addplot[x=Jahr, y=Treffer per Mio. Wörter, color=zhawgray, fill]table{\loadedtable};
        \end{axis} 
        \end{tikzpicture}  
        \caption{Relative Häufigkeit für \texttt{([pos="CARD"]\-[lemma="Lehr\-er|Lehrer\-in"]}}
        \label{figure:5-4} 
    \end{figure}



    \renewcommand{\arraystretch}{1.2}{
    \begin{table}[!ht]
        \footnotesize
        \centering
        \begin{tabular}{p{0.1cm}p{5cm}p{1cm}p{5cm}} \toprule
            \textbf{Position} & \textbf{Kontext vor} & \textbf{Suchbegriff} & \textbf{Kontext nach} \\ \midrule 
            2 & Schenken wir darum schließlich auch diesem Falle noch unsere Aufmerksamkeit ! Unsere  & Schulstatistik & \textbf{zeigt} allerdings , daß verhältnißmäßig mehr Lehrer ihren Stand wechseln , als  \\ 
            4 & zu genügen . Und in der That , ein Blick ans die & Zensurtabellen & \textbf{zeigt} , daß die Schulen um die Stadt herum fast alle nur   \\ 
            7 & im hiesigen Cantone unmittelbar vor der Einführung des nencn Schulgesetzes \textbf{zeigt} in & statistischer & Hinsicht die auf Beilage Nr. VI. befindliche , vom Erzichungsdepartemente damals veranstaltete \\ 
            8 & mehrere Präsidenten dieser Behörden aus . An mehrern Orten , mie die  & Vericht\-erstattungs\-tabelle & \textbf{zeigt} , steht es mit Rücksicht auf den Schulbesuch der Schulpflegen noch   \\ 
            9 & Real-Abtheilung zusammen 264 , und gegenwärtig 384 Schüler ; die Zusammenstellung der & Jahrestabellen & 	
            seit 1856 \textbf{zeigt} , einige Schwankungen abgerechnet , eine stetige Zunahme der   \\ 
            10 & Gemeinden wenig Er » freuliches berichtet werden . In Starkenbach zeigt die & Absenzen\-tabelle & auf einen Schüler durchschnittlich 9Ve , in Wintersberg 12 und in Ennetbühl   \\ 
            \bottomrule
        \end{tabular}
        \caption{KWIC-Ansicht für \textit{zeigt} im Zusammenhang mit \texttt{[word=".*ta\-bell.*|\-.*sta\-tist.*"]}}
        \label{table:5-2}
    \end{table}
}