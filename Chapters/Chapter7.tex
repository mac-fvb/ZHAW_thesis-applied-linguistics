% Indicate the main file. Must go at the beginning of the file.
% !TEX root = ../main.tex

%----------------------------------------------------------------------------------------
% CHAPTER TEMPLATE
%----------------------------------------------------------------------------------------


\chapter{Fazit} % Main chapter title

\label{Chapter7} % Change X to a consecutive number; for referencing this chapter elsewhere, use \ref{ChapterX}

%----------------------------------------------------------------------------------------
% SECTION 1
%----------------------------------------------------------------------------------------

\section{Die Digitalisierung der Schweizer Gesellschaft anhand des Schuldiskurses}

Ziel der Arbeit war es, den Diskurs um die Volksschule in der Schweiz im 19.~Jahrhundert mit korpuslinguistischen Methoden zu analysieren. Dabei sollte herausgearbeitet werden, wie Statistik auch in der Schweiz dazu beitrug, Vorstellungen von Gesellschaft möglich zu machen und wie sich dies diskursiv im vorliegenden Korpus niederschlug. Die Annahme war, dass wir deshalb bereits im 19.~Jahrhundert von einer Digitalisierung der Schweizer Gesellschaft sprechen können, die sich nur durch digitale Techniken beschreiben liess. Diese digitalen Techniken reduzierten die komplexe Gesellschaft der Individuen auf diskrete Zahlen und Kategorien und erlaubten es so, latente Verhaltensmuster sichtbar zu machen.

%-----------------------------------
% SUBSECTION 1
%-----------------------------------
\subsection{Statistikgeschichte}

Zunächst wurde in Kapitel~\ref{Fragestellung} gefragt, ob es beobachtbare Veränderungen auf der sprachlichen Oberfläche des Diskurses gibt, die auf eine zunehmende Bedeutung statistischer Redeweisen gibt. Die explorative Analyse in Kapitel~\ref{Chapter4} konnte nachweisen, dass sich im vorliegenden Korpus ab etwa 1830 zunehmend statistische Termini finden. Tabellen, Statistiken, Durchschnitte und Prozentangaben tauchen auf der sprachlichen Oberfläche des Diskurses auf. Am Wort \textit{Statistik} konnte gezeigt werden, dass es Kompositabildungen anregt. Im Diskurs tauchen Statistiken aller Art auf, etwa zu Schulen, zur Kriminalität oder zum Obstbau. Dass es möglich ist, solche Komposita zu bilden, illustriert, wie die statistische Selbstbeobachtung dazu führt, unterschiedliche Sozialbereiche und Themenfelder abzustecken.

%-----------------------------------
% SUBSECTION 2
%-----------------------------------

\subsection{Statistik im Schuldiskurs}

In Kapitel~\ref{Chapter5} konnten zuerst in Bezug auf Frage 3 gezeigt werden, in welche Gruppen Schüler*innen kategorisiert wurden und wie diese Kategorien als Repräsentation von Gesellschaft gelesen werden können. Es waren dies zunächst Geschlecht und regionale Kriterien wie die Aufteilung in Schulbezirke. Gegen Ende des Untersuchungszeitraums konnte an einem Beispiel eine zunehmende Komplexität belegt werden, die auch Religion und soziodemografischen Status des Vaters mit einbezog. 

Allerdings konnte kaum, wie von Frage 4 gefragt, nachgewiesen werden, dass diese Kategorien gesellschaftlicher Selbstbeschreibung häufig zueinander in Bezug gesetzt wurden. Schüler*innen\-zahl\-en wurden in Tabellen und Listen, meist nach Bezirk oder Ort, rapportiert, aber nur wenige Belege gefunden, wie diese Kategorien diskursiv verbunden wurden, um Muster sozialen Verhaltens zu belegen. Argumente wie beispielsweise «Die Statistik zeigt, dass katholische Kantone weniger Schulen besitzen und ärmer und weniger gebildet sind» wurden explizit nur selten gemacht.

Anhand von Statistiken und Tabellen zu den Schulabsenzen konnte deutlicher gezeigt werden, dass Techniken des «digitalen Sehens» im Schweizer Diskurs um die Schule etabliert waren. Hier konnten zentrale Annahmen bestätigt werden, etwa dass Statistiken dafür eingesetzt wurden, Muster des Verhaltens sichtbar zu machen und zu adressieren. Ausserdem wurde der den digitalen Techniken inhärente Kontrollüberschuss aufgezeigt. Durch die Rekombination der Daten wurde beispielsweise sichtbar gemacht, in welchen Orten die Schulabsenzen höher waren und dies mit der Sozialstruktur der betroffenen Gemeinden in Verbindung gebracht.

\subsection{Doch keine Digitalisierung im 19.~Jahrhundert?}

Die These einer Digitalisierung der Schweizer Gesellschaft zwischen 1830 und 1870 im Spiegel der Schulstatistik kann daher in Ansätzen bestätigt werden. Sie überzeugt vor allem für die allgemeine Entwicklung des Schulwesens und die Durchsetzung der allgemeinen Schulpflicht mittels Absenzenlisten. Mittels der Schul- und Schülerzahlen wurde der Fortschritt des Schulausbaus verfolgt und ansichtig gemacht. Kantonale Darstellungen zeigten, wie es um den regionalen Ausbau der Schulen stand, aber es wurden nur selten konkrete Zusammenhänge zwischen Sozialstruktur und Schulausbau formuliert.

Dies könnte darauf hinweisen, dass das Statistikwesen der Schweiz doch einen Nachzügler im europäischen Vergleich darstellt. Diese These wird von der bisherigen Forschung so vertreten und begründet mit der schwachen Position des Eidgenössischen Bureaus für Statistik gegenüber den Kantonen. Ruoss beispielsweise untersucht differenzierte Schulstatistiken ebenfalls erst ab 1890 (\cite{ruoss_zahlen_2018}). Um diese These genauer überprüfen zu können, müsste das Korpus auf Quellen nach 1870 ausgeweitet werden.

Ein weiterer Grund, weshalb die Hypothese nicht überzeugend bewiesen werden konnte, könnten die Korpusquellen sein. Genügen die im Korpus vertretenen Texte doch nicht für eine entsprechende Untersuchung oder müsste es erweitert werden? Die angewandten Suchtstrategien könnten ebenfalls im Lichte dieser Untersuchung optimiert werden. Es stehen weitere Ansätze der maschinellen Textanalyse zur Verfügung, die hier nicht angewendet wurden. Das gewählte Vorgehen war ausserdem einigermassen eklektisch. 

Ein gewichtiges Problem des Korpus stellt die problematische Datenqualität dar. Wie an mehreren Stellen gezeigt, sind die OCR-erkannten Texte mit vielen Fehlern behaftet, besonders die von Frakturschriften. Zahlreiche Wörter wurden nicht oder falsch erkannt und werden von den Abfragen nicht erfasst, was vor allem induktive Zugänge erschwert. Hier könnte eine auf maschinellem Lernen basierte Texterkennung mit einem für deutschsprachige Fraktur optimierten Modell die Datenqualität deutlich verbessern (\cite{noauthor_german_2022,noauthor_druckwerke_2022}). Als Folge der mangelnden Datenqualität ist die Lemmatisierung und Annotierung des Korpus ungenau. Allerdings kann man davon ausgehen, dass sich die OCR-Fehler als statistischer Noise gleichmässig verteilen und die Aussagen nicht zu stark verzerren. Generell fragt sich, ob das Problem für retrodigitalisierte Quellen in den Griff zu bekommen ist oder die Korpusgrösse so stark ausgeweitet werden muss, damit die OCR-Fehler noch mehr als Noise untergehen. Letztlich schränken diese Probleme aber die Aussagekraft dieser Untersuchung erheblich ein. 

%----------------------------------------------------------------------------------------
% SECTION 2
%----------------------------------------------------------------------------------------

\section{Das Potential der korpuslinguistischen Methode für die Sozialgeschichte}

Kann eine korpuslinguistische Analyse historischer Diskurse das Versprechen einlösen, das die Begriffsgeschichte der 1970er-Jahren der Sozialgeschichte versuchte anzubieten? Programmatisch war der Anspruch der Begriffsgeschichte, mehr zu liefern als eine Ideengeschichte, die den «Höhenkamm» kanonischer Texte abschritt. Busse zufolge ist sie an diesem Vorhaben jedoch gescheitert (\cite[50-71]{busse_historische_1987}, als Entgegnung darauf \cite[306]{dipper_geschichtlichen_2000}). Tatsächlich bietet die Korpuslinguistik die Chance, das begriffsgeschichtliche Arbeiten zu erweitern, indem eine grössere Spannbreite an Texten in die Analyse einbezogen wird. So können Semantisierungsprozesse der untersuchten Begriffe breiter untersucht werden und, sofern die Quellen dafür verfügbar sind, auch Texte aus den «Niederungen» einbezogen werden. Nicht zu unterschätzen ist die Funktion einer maschinellen Suche im Korpus, die Belege aus Texten zutage fördern kann, in denen sie nicht erwartet werden. Das erhöht die Repräsentativität korpusgestützt arbeitender Diskursanalysen.

CQPWeb und andere Tools stellen aber nicht von selber neue Zusammenhänge her. Entscheidend bleibt die Fragestellung, die Modellierung des Diskurses und die Auswahl der Texte. Die resultierenden Ergebnisse von CQPWeb und anderen Tools sind wiederum interpretationsbedürftig (\cite[33]{schwandt_digitale_2018}). Sie müssen anhand der Korpusdaten evaluiert werden und lösen so neue Fragen an das Korpus aus, in einem fortlaufenden Feedback loop (\cite[320-322]{bubenhofer_sprachgebrauchsmuster_2009}).

\section{Limitationen}

Ist die hier verfolgte Methodik einer qualitativen Diskursanalyse überlegen? Nein, vielmehr müssen sich beide Methoden ergänzen. Der Mehrwert des Verfahrens dieser Arbeit besteht darin, dass Befunde quantitativ abgestützt und einzelne Aussagen nicht willkürlich als diskursbeeinflussend beschrieben werden können. Es gibt so eine höhere Überprüfbarkeit. Andererseits braucht es neben der quantitativen Analyse zwingend auch den qualitativen Blick in die Ergebnisse und einen hermeneutischen Verstehensprozess, um die Belege deuten zu können. Das gilt sowohl für die absoluten Frequenzen als auch die Kollokationslisten. Was plausibel und relevant ist, unterliegt immer der Fragestellung und somit auch einem Interpretationsprozess.

Die Fragestellung bestimmt ausserdem massgeblich die Zusammenstellung des Korpus. Weiter hängt die Zusammenstellung des Korpus von der Verfügbarkeit der Quellen ab: Nur ein kleiner Teil der historischen Quellen ist digitalisiert und Ansätze wie die \textit{micro\-storia} oder «Geschichte von unten» haben zurecht darauf hingewiesen, Über\-lieferungs\-zusammen\-hänge ebenfalls in die Quellen\-kritik aufzunehmen. Was überliefert wird, vor allem in gedruckter Form in Archiven und Bibliotheken, sind Dokumente der Elite. Auch diese lassen sich auf die Spuren nicht\-hegemonialer Diskurse lesen (\cite{guha_chandras_1987}). Dies geht aber kaum mit den hier vorgestellten Methode oder anderen Formen des «distant reading». Diese Fragen verschärfen sich, denn die technologischen Voraussetzungen für die Digitalisierung von Texten sind weltweit ungleich verteilt und bevorzugen das, was in (westlichen) Bibliotheken und Archiven überliefert wurde und als bewahrens- und digitalisierungs\-würdig angesehen wird (\cite{putnam_transnational_2016}).

Die in dieser Arbeit angewendeten quantitativen Verfahren sind nicht voraussetzungslos und «naiv». Sie sind statistische Versuche, Signifikanz zu modellieren, aber auch sie gehen von Voraussetzungen aus und treffen Annahmen. Es gibt gar keine Möglichkeit, einen Korpus ohne Vorannahmen zusammenzustellen und diesen komplett objektiv zu untersuchen. Ähnlich wie die Statistiken, die Thema dieser Arbeit sind, sprechen auch Sprachdaten niemals für sich selbst. Sie werden genauso unter gewissen Gesichtspunkten hergestellt und müssen in den Diskurs eingebunden werden.

Quellenkritik ist auch bei digitalen Korpora wichtig. Ramisch zeigt, dass sogar ein redaktionell aufwändig erstelltes Korpus wie die Protokolle des Deutschen Bundestags erhebliche Probleme aufweist (\cite{ramisch_open_2022}). Die Datenqualität ist in dem hier verwendeten Korpus noch weniger zufriedenstellend. Viele Historiker dürften aber mit einem Blick auf die zahlreichen nicht oder falsch erkannten Wörter oder die Vermischung von Paratext und Text in der KWIC-Ansicht die Aussagekraft in Zweifel ziehen und die Vorzüge eines traditionellen hermeneutischen Verfahrens der Diskursanalyse betonen.

\section{Desiderate}

Diese Arbeit konnte die Hypothese einer Digitalisierung der Schweizer Gesellschaft im Spiegel der Schulstatistiken nicht vollständig positiv beantworten. Zu überlegen ist daher, das Korpus um die Jahre 1870 bis 1900 zu erweitern. Ebenfalls fruchtbar könnte es sein, die vermuteten Zusammenhänge zwischen statistischer Beschreibung und Gesellschaftskonzeption auf Ebene einzelner Kantone zu untersuchen. Kantonale Identitäten spielten in der «Erfindung der Schweizer Nation» eine wichtige Rolle und staatliche Zentralisierungsprozesse wurden oft eher mit dem Kanton als dem Bundesstaat identifiziert. Das betraf auch die Schulen. 

In methodischer Hinsicht bieten die derzeit populären Ansätze des \textit{Natural Language Processing} viele Möglichkeiten, um das vorliegende Korpus zu bearbeiten. Sie setzen oft auf maschinelles Lernen und eine datengetriebene Auswertung. Zu denken ist etwa an Methoden wie \textit{topic modeling}, eine \textit{ngram-}Analyse oder \textit{Word embedings}. Diesen Ansätzen ist gemeinsam, dass sie ein beinahe alinguistisches Verständnis von Text haben. Dieser wird als Datensatz aufgefasst, in dem sich statistische Muster finden lassen. Bei diesen Ansätzen stellt sich die Frage, welche Rolle eine hermeneutische Analyse von Sprache noch spielen kann. Allerdings lässt sich mit guten Gründen daran festhalten, dass die Antworten von ChatGPT \string& Co.~nicht von selbst verstanden werden können, sondern nur Ansatz für weitere Fragen sind (\cite{bubenhofer_wenn_2018}).