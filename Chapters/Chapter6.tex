% Indicate the main file. Must go at the beginning of the file.
% !TEX root = ../main.tex

%----------------------------------------------------------------------------------------
% CHAPTER TEMPLATE
%----------------------------------------------------------------------------------------


\chapter{Diskussion} % Main chapter title

\label{Chapter6} % Change X to a consecutive number; for referencing this chapter elsewhere, use \ref{ChapterX}

%----------------------------------------------------------------------------------------
% SECTION 1
%----------------------------------------------------------------------------------------

\section{Digitalisierung der Schweizer Gesellschaft}

Lorem ipsum dolor sit amet, consectetur adipiscing elit. Aliquam ultricies lacinia euismod. Nam tempus risus in dolor rhoncus in interdum enim tincidunt. Donec vel nunc neque. In condimentum ullamcorper quam non consequat. Fusce sagittis tempor feugiat. Fusce magna erat, molestie eu convallis ut, tempus sed arcu. Quisque molestie, ante a tincidunt ullamcorper, sapien enim dignissim lacus, in semper nibh erat lobortis purus. Integer dapibus ligula ac risus convallis pellentesque.

%-----------------------------------
% SUBSECTION 1
%-----------------------------------
\subsection{Bildung gesellschaftlicher Kategorien}

Nunc posuere quam at lectus tristique eu ultrices augue venenatis. Vestibulum ante ipsum primis in faucibus orci luctus et ultrices posuere cubilia Curae; Aliquam erat volutpat. Vivamus sodales tortor eget quam adipiscing in vulputate ante ullamcorper. Sed eros ante, lacinia et sollicitudin et, aliquam sit amet augue. In hac habitasse platea dictumst.

%-----------------------------------
% SUBSECTION 2
%-----------------------------------

\subsection{Problematisierung anhand von Statistiken}
Morbi rutrum odio eget arcu adipiscing sodales. Aenean et purus a est pulvinar pellentesque. Cras in elit neque, quis varius elit. Phasellus fringilla, nibh eu tempus venenatis, dolor elit posuere quam, quis adipiscing urna leo nec orci. Sed nec nulla auctor odio aliquet consequat. Ut nec nulla in ante ullamcorper aliquam at sed dolor. Phasellus fermentum magna in augue gravida cursus. Cras sed pretium lorem. Pellentesque eget ornare odio. Proin accumsan, massa viverra cursus pharetra, ipsum nisi lobortis velit, a malesuada dolor lorem eu nequ

%----------------------------------------------------------------------------------------
% SECTION 2
%----------------------------------------------------------------------------------------

\subsection{Kontrollüberschuss der digitalen Methoden}


\section{Zahlen erzählen: zum diskursiven Verweis auf Statistiken und Statistik als Argument}

\begin{itemize}
    \item\cite{espeland_narrating_2015} hat einen spannenden Ansatz dafür, dass Zahlen nicht für sich selber sprechen, sondern diskursiv zum Thema gemacht werden müssen, sie müssen in Narrative eingebettet werden
    \item Zahlen werden gemäss Heintz eingesetzt, «Argumente mit einer Aura des Notwendigen zu versehen»
    \item Zahlen, die empirische Fakten abbilden sollen, haben eine spezielle argumentative Qualität: sie erlauben, ähnlich wie Abbildungen, keine einfache Negation in Form eines "Ja/Neins", sondern ihre Herstellung muss problematisiert werden, das ist aufwändiger als die rein sprachliche Negation \cite[78]{heintz_zahlen_2007} 
    \item \cite{ruoss_zahlen_2018} hat dazu auch für den Fall der Schweizer Bildungsgeschichte etwas gemacht, sehr aufschlussreich, aber erst ab 1890
    \item Wir fragen hier, ob es auf der sprachlichen Oberfläche Hinweise gibt, wie Zahlen und Statistiken argumentativ eingesetzt werden. 
\end{itemize}

\begin{verbatim}
    [word ="zahl.*|.*statist.*"] []* [pos ="VVFIN"] within s
\end{verbatim}