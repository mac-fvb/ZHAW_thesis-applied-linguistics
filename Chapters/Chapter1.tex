% Indicate the main file. Must go at the beginning of the file.
% !TEX root = ../main.tex

%----------------------------------------------------------------------------------------
% CHAPTER 1
%----------------------------------------------------------------------------------------



\chapter{Einleitung} % Main chapter title

\label{Chapter1} % For referencing the chapter elsewhere, use \ref{Chapter1} 

%----------------------------------------------------------------------------------------


%----------------------------------------------------------------------------------------
\section{Hypothese: Digitalisierung im 19.~Jahrhundert?}
Die Digitalisierung der Schweiz beginnt im frühen 19.~Jahrhundert. Diese These werde ich in der Arbeit mit einer korpuslinguistischen Untersuchung des deutschschweizerischen Bildungsdiskurses der 1830er bis in die 1870er-Jahre untersuchen.\footnote{Für die Recherche des Forschungsstandes zu dieser Arbeit wurde die generative KI \Citetitle{openai_chatgpt_2023} eingesetzt. Es wurden keine sinngemässen oder wörtlichen Zitate von ChatGPT entnommen.}

Ab den 1830er-Jahren intensiviert sich in der Schweiz, wie auch in anderen Ländern Europas, ein Diskurs um die Organisation und Verbesserung der Schulbildung, der in der Spätaufklärung wurzelt. Privatpersonen, Lehrer, Geistliche, liberale und konservative Politiker\footnote{Hier liegt kein generisches Maskulinum vor. Die Diskursteilnehmer sind, soweit ich es überblicken kann, fast ausschliesslich männlich.} diskutieren die Rolle der Schule, vom Verhältnis religiöser und beruflicher Bildung über die Lehrerbildung bis zur Mädchenbildung. Dieser Diskurs findet zuvorderst in Zeitschriften und Publikationen statt und er wird vermehrt geprägt von Statistiken. Denn der Untersuchungszeitraum fällt zusammen mit dem Aufkommen der Sozialstatistik als zentralem Mittel gesellschaftlicher Selbstbeschreibung. Eine «avalanche of printed numbers» setzte um 1820 ein, während der die Anzahl an veröffentlichten Statistiken rapide anstieg (\cite{hacking_biopower_1982}). Die Akteure, die diese Flut an gedruckten Zahlen verursachten, überschneiden sich mit den Akteuren des bildungsreformerischen Diskurses. So war der spätere Bundesrat Stefano Franscini sowohl Autor einer der ersten statistischen Beschreibungen der Schweiz wie auch Lehrer und als Politiker der Motor kantonaler und föderaler Bildungsreformen (\cite{marcacci_franscini_2022}). 

Dass dieser Diskurs öffentlich geführt wird und Statistiken darin eine wichtige Rolle spielen, ist eine Funktion der Digitalisierung der Schweizer Gesellschaft. Die enge Verbindung von Schulreform und Statistik ist kein Zufall, sondern hat gemeinsame Wurzeln in den politisch-sozialen Umbrüchen des frühen 19.~Jahrhunderts, zuvorderst der Durchsetzung des «modernen» Staates mit dem liberalen Ideal rechtlich gleichgestellter Staatsbürger und einer Ausdifferenzierung der Gesellschaft infolge der Frühindustrialisierung. Mit Armin Nassehi gehe ich davon aus, dass die Entwicklung der sozialkundlichen Statistik seit dem frühen 19.~Jahrhundert als Funktion einer Digitalisierung der Gesellschaft beschrieben werden kann. Ihm zufolge stehen die aufkommenden Nationalstaaten vor einem Problem: Das Ende der Ständestaaten erzeugte die Vorstellung von territorial definierten Staaten, in denen gleichberechtigte Bürger zusammenlebten. Diese Bevölkerung war nur mit «digitalen» Methoden beschreib-, erfass- und steuerbar: diskreten, frei re-kombinierbaren Signifikatoren, die «reiche» soziale Phänomene auf Zahlen reduzierten, die miteinander auf vielfältige Weise in Bezug gesetzt werden konnten (\cite[63]{nassehi_muster_2019}). 

Digitalisierung wird also funktionalistisch verstanden. Sie ist eine Art und Weise der gesellschaftlichen Selbstbeschreibung, die auf die Anforderungen der modernen Gesellschaft reagiert. Konkret zeigt sich Digitalisierung der Gesellschaft am Prozess der Herstellung solcher abstrakter Signifikatoren. Diese Signifikatoren werden in öffentlichen Diskursen verhandelt und zur Selbstbeschreibung einer Gesellschaft herangezogen: Die unübersichtliche Gesellschaft wird beschreibbar als Zusammenspiel der aggregierten Kategorien, die Regelmässigkeiten entdecken lässt. Für den vorliegenden Fall lässt sich das anschaulich machen. Schulen, Schülerinnen und Schüler, Lehrerinnen und Lehrer werden gezählt. Dazu müssen sie in Kategorien eingeteilt werden, die zueinander in Bezug gesetzt werden können. Das erlaubt, gemeinsame Muster im Verhalten der Individuen zu beschreiben. Die unübersehbare Vielzahl der Individuen wird intellektuell und sprachlich handhabbar und als Gesellschaft beschreibbar. Dieser Prozess schlägt sich diskursiv nieder, was diese Arbeit anhand einer quantitativ-qualitativen Analyse der bildungsreformerischen Diskurse der deutschsprachigen Schweiz im 19.~Jahrhundert versuchen wird zu zeigen.

\section{Untersuchungsgegenstand und Studiendesign}

Untersucht wird ein bereits bestehendes Korpus aus gedruckten Quellen, die zwischen dem 1.~Januar 1800 und dem 31.~Dezember 1870 publiziert wurden und das mit den Suchbegriffen \textit{schulbildung}, \textit{schulwesen}, \textit{kantonalschul}, \textit{volksschul}, \textit{primarschul} aus Digitalisaten der Plattformen \href{https://www.e-rara.ch/}{e-rara.ch} und \href{https://e-periodica.ch}{e-periodica.ch} zusammengestellt wurde. 

Die Arbeit wird in einem ersten Schritt dieses Korpus mit quantitativen Methoden der Korpuslinguistik auf statistisch signifikante Phänomene auf der «Textoberfläche» untersuchen. Das heisst, es wird nach signifikanten Zeichenfolgen gesucht und zunächst versucht zu vermeiden, diese semantisch zu bestimmen. Die Ergebnisse dieser ersten statistischen Exploration sollen dazu dienen, die Abfragen zielgerichteter auf bestimmte sprachliche Muster einzuschränken. Die explorative Untersuchung liefert darüber hinaus Belegstellen für qualitative Betrachtungen, ob sie auch auf semantischer und inhaltlicher Ebene die angenommenen diskursiven Entwicklungen bestätigen. Anschliessend werden die Ergebnisse der Exploration ergänzt um hypothesengestützte Analysen, die versuchen aufzuzeigen, wie Schulstatistiken eine Vorstellung von Gesellschaft erzeugen.

Dieser Ansatz entspricht einem \textit{mixed-methods-}Verfahren mit zwei parallelen Strängen, einem eher quantitativen und \textit{corpus-driven} Schritt und einem parallelen, aber textlich anschliessenden \textit{corpus-based} Schritt. Eine Triangulation findet dadurch sowohl auf erkenntnistheoretischer Ebene (Erklären vs.~Verstehen) als auch auf methodischer Ebene (statistische Analyse vs.~Hermeneutik) statt. Durch die multiperspektivische Analysen, die weitere Analysen informieren, soll die Aussagekraft der Ergebnisse verbessert werden (\cite[139]{dreesen_diskurslinguistik_2019}).

\pagebreak

\section{Fragestellung}\label{Fragestellung}

Der besprochene Korpus wird im Hinblick auf folgende Fragen untersucht:

\begin{enumerate}
    \item Gibt es beobachtbare Veränderungen auf der sprachlichen Oberfläche des Diskurses, die darauf hinweisen, dass die sozialkundliche Statistik Einfluss auf die gesellschaftliche Selbstbeschreibung gewinnt?
    \item Werden Fragen und Probleme des bildungskundlichen Diskurses quantifiziert?
    \item Welche Kategorien werden für die Quantifizierung eingesetzt?
    \item Wie werden diese Kategorien sprachlich zueinander in Bezug gesetzt? 
\end{enumerate}

\section{Relevanz für die Angewandte Linguistik}

Diese Arbeit ist auf den ersten Blick nicht der Angewandten Linguistik im engeren Sinn zuzuordnen. Sie entspricht in Anlage und Erkenntnisinteresse mehr den Digital Humanities. Der Bezug zur Linguistik liegt in der verwendeten Methodik: Die Arbeit setzt auf Methoden der Korpuslinguistik und der Untersuchung kultur- und sozialgeschichtlicher Phänomene, die sich in den letzten Jahren etabliert haben (\cite{bubenhofer_sprachgebrauchsmuster_2009}). Die im Folgenden aufgezeigten und umgesetzten Arbeitsschritte der Diskursmodellierung und -analyse stammen aus der Angewandten Linguistik und lassen sich auch auf Fragestellungen anwenden, wie sie beispielsweise in der Organisationskommunikation vorkommen, auch wenn sie hier an einem historischen Thema vorgeführt werden. Das theoretische Wissen über den Sprachgebrauch in Diskursen ermöglicht einen reflektierteren Sprachgebrauch. Im Berufsalltag ist Wissen über die vorherrschenden Formulierungen und Themen eines Diskurses relevant, denn es ermöglicht eine anschlussfähige Kommunikation. Die theoretischen Überlegungen zu Diskursen und ihre Untersuchungsmöglichkeiten mit korpuslinguistischen Diskursen sind daher berufsrelevant. Diskurse lassen sich zwar nicht zielgerichtet manipulieren, aber eine empirisch gestützte Untersuchung liefert mögliche Optionen: Welche sprachliche Äusserung ist innerhalb eines Diskurses anschlussfähig, welche Akteure können eher als andere erreicht werden (\cite{dreesen_diskurslinguistik_2019})?

\section{Abgrenzung}

Das Erkenntnisinteresse der Arbeit ist primär ein sozialhistorischer Blick auf den Diskurs über die Statistik, wie er in den bildungskundlichen Medien der Deutschschweiz im 19.~Jahrhundert geführt wurde und welche Effekte er auf gesellschaftliche Selbstbeschreibungen hatte. Auch wenn dafür die Geschichte des schweizerischen Volksschulwesens und Kenntnisse der Geschichte der Statistik unabdingbar sind, stehen sie nicht im Fokus. Da hier Analysemethoden der Angewandten Linguistik auf einen bisher kaum untersuchten Gegenstand angewendet werden, stehen inhaltliche Erkenntnisse zum historischen Phänomen im Vordergrund. Dagegen geht es weniger um eine methodisch-theoretische Weiterentwicklung dieser Forschungsansätze oder um Probleme der Angewandten Linguistik.

\section{Struktur der Arbeit}

Im folgenden Kapitel~\ref{Chapter2} wird zunächst der historische Kontext des Untersuchungsgegenstandes aufgearbeitet, die Geschichte des Schweizer Volksschulwesens und ihr enger Zusammenhang mit der Entstehung des modernen Staates, dessen Bewohner als einer Gesellschaft zugehörig verstanden werden und welche Bedeutung der Statistik darin zukam. Kapitel~\ref{Chapter3} skizziert den theoretischen Hintergrund der Diskursanalyse und Korpuslinguistik. Daran schliesst der Versuch einer induktiven Analyse des Korpus in Kapitel~\ref{Chapter4} an. In Kapitel~\ref{Chapter5} werden die Ansätze der induktiven Analyse weiterverfolgt und mit einem hypothesen-basierten Verfahren ergänzt. Ein kritisches Fazit~\ref{Chapter7} sowohl der erzielten Ergebnisse als auch der Methodik schliesst die Arbeit ab. 