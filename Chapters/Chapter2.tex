% Indicate the main file. Must go at the beginning of the file.
% !TEX root = ../main.tex

%----------------------------------------------------------------------------------------
% CHAPTER TEMPLATE
%----------------------------------------------------------------------------------------


\chapter{Die Entdeckung der Gesellschaft: Statistik, Bildungsreform und Nationsbildung im 19. Jahrhundert} % Main chapter title
\label{Chapter2} % Change X to a consecutive number; for referencing this chapter elsewhere, use \ref{ChapterX}

%----------------------------------------------------------------------------------------
% SECTION 1
%----------------------------------------------------------------------------------------

\section{Die Schweiz in den transnationalen Diskursen um Bildung und Statistik}

\subsection{Die Bildungsexpansion in der Schweiz, 1770–1830}

Im letzten Drittel des 18.~Jahrhunderts lässt sich in Europa eine Transformation des Bildungswesens beobachten. Nicht nur nimmt die private Nachfrage nach Bildungsangeboten deutlich zu, sondern gleichzeitig interessieren sich auch die Regierungen dafür, was in den bislang religiös geführten Bildungseinrichtungen geschieht. Heinrich Busse sieht hier einen doppelten Prozess: eine Revolution von unten, bei der Bildung einen sozialen Aufstieg ermöglicht und deshalb von den Akteur*innen immer mehr nachgefragt wird sowie eine Kulturrevolution von oben, in der die aufgeklärte Bürokratie das Schulwesen rational vereinheitlichen und «zukunftsfähig» machen will (\cite[50, 59]{bosse_bildungsrevolution_2012}).  

Auch in der Schweiz lässt sich dieser Doppelprozess nachweisen. Um 1770 intensiviert sich das Interesse an der Schulbildung, besonders an den Landschulen. Auch hier gibt es spätaufklärerische Stimmen, die ihre Ideen in der Landbevölkerung verbreiten möchten und eine Verbesserung der Bildung fordern. Ebenso regulieren die Obrigkeiten nun das Schulwesen stärker. Im Kanton Zürich wird 1771/1772 erstmals eine Schulumfrage aller Landschulen durch Private organisiert, 1778 erlässt die Zürcher Kantonsregierung eine Landschulordnung (\cite[48-63]{bloch_pfister_priester_2007}). In der Stadt Zürich entstehen parallel dazu neue staatliche und private Bildungseinrichtungen. War höhere Bildung bisher vor allem religiös verstanden und zielte auf die Ausbildung von Pastoren gibt es nun neue Lehranstalten, die juristische, handwerkliche oder kaufmännische Fähigkeiten vermitteln (\cite[156-157]{trohler_classical_2011}).

Träger dieses Prozesses von unten waren Anhänger der Aufklärung, die sich in sozial- und moralreformerischen Vereinen wie der Helvetischen Gesellschaft versammeln. Einflussreich waren etwa Heinrich Zschokke oder Johann Heinrich Pestalozzi, der seine Pädagogik in mehreren Reformschulen zu vermitteln suchte (\cite{graf_pestalozzi_2022}, \cite[68]{butikofer_staat_2006}).

\pagebreak

Dieser Ausbau des schweizerischen Bildungswesens war massgeblich geprägt vom politischen Grosskonflikt der Zeit zwischen, vereinfacht zusammengefasst, liberalen Anhängern eines säkularisierten, demokratischen Bundesstaats mit Volkssouveränität und Volksrechten auf der einen Seite und auf der anderen Seite konservativen Vertretern, die auf der Souveränität der Kantone und dem Erhalt ständischer Privilegien und einer verankerten Rolle der Religion im Staatswesen beharrten.

In der Helvetischen Republik von 1798 bis 1803 erlebte die Bildungs- und Schulreformen in der Schweiz einen ersten Durchbruch. Die nach dem Einmarsch französischer Revolutionstruppen 1798 geschaffene Republik ersetzt die vorher nur lose verbundenen Kantone durch einen Einheitsstaat. Nach französischem Vorbild geht das Erziehungswesen in die Hand eines zentralen staatlichen Erziehungsdepartements, dessen Vorsteher Johann Heinrich Stapfer eine Schulreform anstrebt. Erstmals wird Schweizweit eine allgemeine Schulpflicht eingeführt. Das Schulwesen wird ganz unter die Aufsicht des Staates und nicht mehr der Kirchen gestellt. Die Schulformen und die Lehrerausbildung sollen vereinheitlicht und professionalisiert werden (\cite{butikofer_staat_2006}). 

Die Helvetische Republik organisiert zu diesem Zweck die erste landesweite öffentliche Schulumfrage der Schweiz, die heute als Stapfer-Enquête bekannt ist. Im Januar 1799 verschickt das Departement einen Fragebogen an sämtliche Lehrer der Schweiz, von dem heute 2400 handschriftliche Antwortbögen überliefert sind. In 16 Fragen mit mehreren Unterfragen werden Angaben zu den Schulen erfragt, von der Schulträgerschaft über die Grösse der Schule und den Schulbesuch, die Lehrinhalte, die Unterrichtsdauer, die Situation der Lehrerschaft und die wirtschaftliche Situation der Schule (\cite{schmidt_stapfer-enquete_2015}, \cite{trohler_volksschule_2014}). 

Die Stapfer-Enquête und weitere kantonale Schulumfragen der Epoche verweisen auf das frühe Interesse in der Schweiz nach dem Zustand der Volksbildung und die Bedeutung, die einer systematischen empirischen Datenerhebung zugemessen wurde. Sie zeigen, dass auch in der Schweiz die sich entwickelnden Methoden der Statistik rezipiert und erprobt wurden, bereits vor Beginn der Helvetik (\cite{rothen_vergessenen_2014}). Die vorwiegend qualitativen und differenzierten Antworten vieler beantworteter Fragebögen der Stapfer-Enquête zeigen aber auch, dass das resolute Kategorisieren und Reduzieren sozialer Phänomene auf eindeutige Zahlenwerte wenig ausgeprägt war (\cref{table:2-1}).

Mit Ausnahme der Einsetzung kantonaler Schul- und Erziehungsräte wurden die weitgehenden Reformpläne der Helvetischen Republik kaum umgesetzt. Mit dem Ende der Republik 1803 wurde die Schulpolitik rekantonalisiert und der Einfluss Geistlicher auf die Schulen nahm wieder zu. Allerdings wiesen die Tendenzen weiterhin in Richtung eines Ausbaus des Volksschulwesens. Darin waren sich Konservative wie Liberale einig, differierten aber in Umsetzungs- und Detailfragen wie dem Einfluss der Kirche oder der Lehrerbildung (\cite[233-234]{butikofer_staat_2006}). Auf kantonaler Ebene lassen sich denn auch weitere Entwicklungen hin zu einem umfassenden Volksschulwesen im frühen 19. Jahrhundert ausmachen. Besonders in neu entstandenen Kantonen wie der Waadt sollte die Schule zur Identitätsbildung einer «kantonalen Nationalkultur» beitragen (\cite{rothenbuhler_neue_2010}). Die Kantone sind, vor dem erst 1848 gegründeten Bundesstaat, ein wichtiger Referenzpunkt für Vorstellungen von Staat und Gesellschaft in der Schweiz im 19.~Jahrhundert. Etwa wurde das Schulhaus in ländlichen Gegenden oft als Symbol für eine unerwünschte kantonale Zentralmacht über die Gemeinden gesehen (\cite[16]{criblez_einleitung_1999}).

\hspace{1cm}

\begin{table}[!ht]
    \centering
    \begin{tabular}{lp{3cm}p{8cm}}
        \toprule
        III.12  & \multicolumn{2}{l}{Schulkinder. Wie viele Kinder besuchen überhaupt die Schule?} \\ 
        \midrule
        III.12.a & Im Winter. (Knaben/Mädchen) & 50. Schulkinder überhaupt besuchen die Schule. im Winter. Namlich. 28. Knaben. 22. Mädchen. \\ 
        III.12.b & Im Sommer. (Knaben/Mädchen) & Jm Sommer ist es sehr ungleich wegen der anzahl der Schul- Kinder. es werden Wochentlich 3. mahl Catichismus übungen gehalten, namlich. Mitwochen Samstag, und Sonntag. Zu wünschen were es das die Kinder bey diesen gesezten Stunden, sich zahlreicher einfinden würden. \\
        \bottomrule
    \end{tabular}
    \caption{Beispielantwort der Stapfer-Enquête für Nr.~165, Oberembrach, \cite{schmidt_stapfer-enquete_2015}. Verfügbar 12.~Februar 2023 unter \url{http://www.stapferenquete.ch/db/165}.}
    \label{table:2-1}
\end{table}

\subsection{Konfliktlinien des schweizerischen Volksschulwesens, 1830–1870}

Als eigentlicher Durchbruch eines staatlichen Schulwesens mit Schulpflicht in der Schweiz gilt der Forschung die Zeit um 1830, als liberale Bewegungen sich in zahlreichen Kantonen durchsetzen und neue Kantonsverfassungen erringen konnten. Dort kam es in der Folge zu neuen Schulgesetzen und dem Versuch, das Volksschulwesen auszubauen, zu professionalisieren und einen allgemeinen Schulbesuch zu verordnen (siehe die Beiträge in \cite{criblez_schule_1999}).

Die Liberalen sahen in der Volksbildung eine wichtige staatliche Aufgabe, auch um den Einfluss von Konservativen und Religiösen auf die Bildung zurückzudrängen. Die Durchsetzung der Schulpflicht blieb im 19.~Jahrhundert einer der massgeblichen Konfliktlinien zwischen liberalen Schulreformern sowie Schulbehörden auf der einen Seite und Teilen der Landbevölkerung, Geistlichen und Konservativen auf der anderen Seite. Für eine Zurückhaltung gegenüber dem Schulbesuch gab es aus Sicht der Landbevölkerung wirtschaftliche wie auch kulturelle Gründe: Kinder älter als 10 Jahre wurden vor allem im Sommer als landwirtschaftliche Arbeitskräfte benötigt und der wirtschaftliche Wert einer Schulbildung erschien den Erziehungsberechtigten oft zweifelhaft. Für Bildungs- und Sozialreformer wurde mangelhafter Schulbesuch zu einer Chiffre für die Rückständigkeit gesellschaftlicher Gruppen und damit zur Legitimation von Interventionen (\cite[602]{dodde_von_1991}). 

Im 1848 neu gegründeten Bundesstaat blieb das Schulwesen in der Hand der Kantone. Dass die Schulbildung in staatlicher Hand liegen sollte und eine allgemeine Schulpflicht bestand, war zu diesem Zeitpunkt auch unter Konservativen nicht mehr bestritten. Die revidierte Bundesverfassung von 1874 konnte das bis dahin in den Kantonen erreichte festschreiben: die Pflicht zum Schulbesuch bis zur Primarstufe in staatlichen und konfessionell neutralen öffentlichen Schulen (\cite{grunder_primarschule_2012}). Weitere Diskussionen betrafen mehr die innere Gestaltung der Schulen, den Ausbau der Schulformen und der sekundären Bildung, die Armen- und Waisenbildung sowie den Mädchenunterricht (siehe die Tabelle der von der Schweizerischen Gemeinnützigen Gesellschaft besprochenen pädagogischen Themen bei \cite[173]{criblez_schweizerische_2013}). Um 1870 kann daher die staatliche Volksschule als in der Schweiz breit akzeptierter Normalfall gelten (\cite[167]{trohler_classical_2011}). 

\section{«The avalanche of numbers»: Statistikgeschichte}

Im 19.~Jahrhundert wurde die Statistik zum massgeblichen Modus gesellschaftlicher Selbstbeobachtung und Selbstbeschreibung, was sich parallel zur Entwicklung des modernen Staates und dem Aufbau eines staatlichen Schulwesens vollzog. Erst das 19.~Jahrhundert dachte in \textit{nationalen} Gesellschaften, deren Wachstum, Reichtum oder Entwicklung mit Zahlen beschrieben werden konnte (\cite[57-62]{osterhammel_verwandlung_2010}).

Vorläufer waren im deutschsprachigen Raum die frühneuzeitlichen Polizeywissenschaften. Michel Foucault macht im 17.~Jahrhundert einen Umbruch in den europäischen Regierungsstilen aus, indem für den Staat die genaue Kenntnis des Landes, seiner Bevölkerung und seiner Ressourcen ins Zentrum der Regierungstätigkeit rückt und die Regierungen Informationen zur Stärke ihres Landes sammelten. Als wichtiger Faktor zur Einschätzung staatlicher Macht war dieses Wissen geheim. Es konnte also nicht öffentlich diskutiert werden und damit gesellschaftliche Selbstbeschreibung prägen (\cite[395-400]{foucault_geschichte_2004}). 

Im 18.~Jahrhundert begannen sich Private und Gelehrte für den Zustand der Staaten, der Wirtschaft oder der Bevölkerung zu interessieren. Der Begriff Statistik wurde 1749 geprägt, hatte aber damals noch keinen exklusiven Bezug zu Zahlen oder einer tabellarischen Darstellung, die wir heute mit Statistiken verbinden. Vielmehr präsentierten die statistischen Länderbeschreibungen eine Mischung geografischer, historischer und anderer Informationen meist in narrativer Form, um etwa eine Region zu beschreiben. Erst zu Beginn des 19.~Jahrhunderts setzte sich allmählich die Vorstellung der Statistik als numerischer Wissenschaft durch (\cite[16-26]{hacking_taming_1990}). 

Die Mathematisierung der Statistik begann aus dem Interesse an der Bevölkerungszahl. In England analysierten Pioniere der Volkszählungen Pfarreiregister und leiteten daraus Geburts- und Sterberaten ab, die fundiertere Schätzungen der Bevölkerung und ihrer Entwicklung ermöglichten. Um 1800 begannen in Grossbritannien und Frankreich staatlich unterstützte statistische Datenerhebungen zu Bevölkerungszahlen, Gesundheitsstatistiken und Zahlen zu Institutionen wie Gefängnissen. Sie entdeckten, dass sich soziale Phänomene nicht zufällig ereigneten, sondern überindividuelle Muster im Verhalten zeigten, obwohl die Entscheidungen den einzelnen Individuen frei erschienen. Beispielsweise war nicht nur die Zahl der Suizide in Gesellschaften über die Jahre hinweg bemerkenswert konstant, die Zahlen schwankten auch regelmässig im Jahresverlauf. Sogar die Arten des Suizids wiesen stabile Muster auf. Diese Regelmässigkeit erlaubte die Annahme, dass «Gesellschaft» ein stabiles Beobachtungsobjekt bildete, dessen Muster Einsichten in die Ursachen individuellen Verhaltens ermöglichte (\cite[75-76]{hacking_taming_1990}). «Soziale Probleme» wurden entdeckt, über die Statistiken Aufschluss geben konnten, zum Beispiel anhand von Kriminalitätsraten oder Sterblichkeitsursachen (\cite[27-31]{porter_rise_1986}).

Aus dieser Erkenntnis folgte ein «enthusiastisches Zeitalter» der Statistik: Europäische Staaten, gemeinnützige Gesellschaften und Sozialreformer erhoben Zahlen zu allen Arten sozialer Phänomene, welche breit rezipiert wurden. Ian Hacking bezeichnet diese Phase als «avalanche of printed numbers», während der von 1820 bis 1840 die Anzahl gedruckter Zahlen  exponentiell gestiegen sei (\cite[282]{hacking_biopower_1982}). Grundlegend ist für Hacking eine Transformation des Verständnisses dieser Zahlen. Für die Zeitgenossen bilden sie so etwas wie Naturgesetze sozialen Verhaltens ab, die Ansatzpunkte für soziale Interventionen sind. Damit die Dinge zählbar werden, müssen sie kategorisiert werden. Die Kategorisierung, Zählung und anschliessende Publikation dieser Zahlen hatte wiederum entscheidende Implikationen für das Selbstverständnis der so beschriebenen Gesellschaft. Die Beschreibung einer Gesellschaft als Klassengesellschaft etwa begünstigt die Identifikation der so Beschriebenen mit einer ebendieser Klasse (\cite{hacking_biopower_1982}).

\subsection{Öffentliche Statistik als Funktion einer Digitalisierung}

Für Armin Nassehi ist die Popularität der öffentlichen Sozialstatistik die «erste Entdeckung der Gesellschaft» und eine Digitalisierung der Gesellschaft. Digitalisierung meint hier, dass die Gesellschaft lernt, sich selbst «digital zu sehen». Digital sehen, das bedeutet Nassehi zufolge, dass reiche, analoge soziale Phänomene auf eine Form, hier Zahlen, reduziert werden, die einen Vergleich und eine Rekombination ermöglichen und damit latente Muster aufdecken und vorhersehbar machen. Das öffentliche Teilen und diskutieren dieser Zahlen ermöglicht ein gesellschaftlich geteiltes Verständnis der Gesellschaft, die sich als Ensemble dieser Kategorien begreift (\cite[32-36]{nassehi_muster_2019}).

Die europäischen Gesellschaften des frühen 19.~Jahrhunderts sind mit einem Problem konfrontiert, für das Digitalität eine Lösung oder zumindest Erklärungen anbietet. Liess sich früher im \textit{Ancien Régime} aus der ständischen Geburt meist zuverlässig auf Welt- und Wertvorstellungen sowie das Schicksal eines Menschen schliessen, sind die Bürger in den neuen Nationalstaaten nominell frei, ihr Leben zu gestalten. Wie die Statistiker jedoch erstaunt feststellten, treffen sie ihre Entscheidungen zu heiraten oder sich selbst umzubringen mit einiger Regelmässigkeit (\cite[44-47]{nassehi_muster_2019}). Können wir in unserem vorliegenden Korpus nachweisen, dass die Teilnehmer des Deutschschweizer Diskurses zur Volksschulbildung lernen, «digital zu sehen»?

\subsection{Schweizer Statistik}

Während der Forschungsstand zur Geschichte des Schweizer Schulwesens umfassend ist, ist der Forschungsstand zur Geschichte der Statistik in der Schweiz wesentlich dünner. Ähnlich wie bei der Bildungsgeschichte war aber auch hier das Gebiet der heutigen Schweiz in die europäischen Diskurse zur Entwicklung der Statistik eingebunden. Daher könnte diese Arbeit einen Beitrag zum Statistikwesen in der Schweiz von 1820 bis 1870 leisten.

Die ältere staatswissenschaftlich-qualitative Statistik wurde im Gebiet der heutigen Schweiz während des 17.~und 18.~Jahrhunderts beispielsweise auch in den Kantonen Bern oder Zürich betrieben (\cite{pfister_uss_1995}). Auch diese ältere beschreibende Statistik interessierte sich für die Schulen, wie die Zürcher Schulumfrage von 1770 zeigt (\cite[50-55]{bloch_pfister_priester_2007}). Die schon erwähnte Stapfer-Enquête zum Zustand der Schulen war nur eine von mehreren Untersuchungen, die die Helvetische Republik ab 1799 initiiert hatte, um den Zustand des Landes und mögliche Probleme in Erfahrung zu bringen. Damit folgte sie ganz dem französischen Beispiel, wo der post-revolutionäre Staat statistische Büros eingerichtet hatte (\cite{holenstein_reform_2014}).

Zu einer Einrichtung eines statistischen Büros für die ganze Schweiz kam es während der Helvetik jedoch nicht. Dafür war die Helvetische Republik zu kurzlebig und die Zentralregierung zu schwach. Gängig ist deshalb das Urteil, die Schweizer Statistik habe nach der Helvetik den Anschluss an europäische Entwicklungen verloren, also gerade in der Phase von 1820 bis 1850, während der sich die Sozialstatistik etablierte (\cite[17-22]{jost_von_2016}). Dieses Urteil fällt meist mit Blick auf die sehr späte Etablierung eines bundesweiten Eidgenössischen Statistischen Büros, dem heutigen Bundesamt für Statistik. Dieses wurde erst 1860 eingerichtet, also zwölf Jahre nach der Gründung des Schweizerischen Bundesstaates 1848 (\cite[25]{jost_von_2016}).

Einzelne Kantone und Private betrieben aber durchaus statistische Erhebungen, die sich an sozialreformerischen Statistikbewegungen andernorts orientierte. Für die Liberalen waren der Aufbau eines staatlich dominierten Volksschulwesens und die öffentliche Statistik miteinander verbundene Projekte. So überschnitten sich beispielsweise die Personenkreise, die sich in beiden Diskursen engagierten. Beispiele dafür sind etwa Stefano Franscini. Nach seiner Tätigkeit als Lehrer wandte er sich der Statistik zu. 1828 publizierte er auf Italienisch eine der ersten statistischen Beschreibungen der Schweiz und war gleichzeitig in Vereinen zur Volksbildung engagiert. Als Mitglied des ersten Bundesrates ab 1848 war er als Leiter des Departements des Inneren für die Statistik zuständig und versuchte 1850, eine erste Volkszählung der Schweiz zu lancieren (\cite{marcacci_franscini_2022}). Eine wichtige Plattform stellte die Schweizerische Gemeinnützige Gesellschaft dar. Sie bot sowohl ein Forum für Diskussionen über das Schulwesen als auch der Statistik bot. Gegen Ende des 19.~Jahrhunderts hatte sie den Anspruch, durch ihre Aktivitäten eine Akteurin in der «Erfindung der Schweizer Nation» zu sein (\cite{criblez_schweizerische_2013,rothenbuhler_neue_2010}).

\subsection{Staat, Schule und Statistik}

Die enge Verbindung zwischen der Verstaatlichung der Erziehung und der Entstehung der modernen Sozialstatistik, wie sie sich seit Beginn des 19.~Jahrhunderts zeigt, ist kein Zufall, sondern sie gehören zum gleichen Prozess: der Entstehung des modernen (National-)Staates. Der moderne Staat unterscheidet sich vom vorangehenden Ständestaat, dass auf seinem Gebiet nominell gleichberechtigte Bürger leben. Auch die Kantone der alten Eidgenossenschaft wiesen ständische Züge auf. Träger des vormodernen Staates waren die patrizischen Oberschichten in Städten wie Bern, Zürich oder Genf sowie in den Landkantonen. Ausserhalb der Hauptorte der Stadtkantone sowie in den gemeinen Herrschaften lebten Untertanen mit unterschiedlichen Rechtsstatus (\cite[137-156]{maissen_geschichte_2011}). 

Die Vorstellung, dass die auf einem Gebiet lebenden Personen durch eine gemeinsame Staatsbürgerschaft verbunden seien, setzte sich erst Ende im 19.~Jahrhundert. Die Stiftung dieser «vorgestellten Gemeinschaften» bedurfte intensiver intellektueller Arbeit, etwa durch das Schreiben einer Nationalgeschichte, der Vereinheitlichung der lokalen Dialekte zur einer Hochsprache oder durch eine Kartografierung der Nation, die eine Imaginierung des Landes als zusammengehörigen Raum ermöglichte (\cite{anderson_imagined_1991}). Die Statistik, die in ihren älteren Formen einer Landesbeschreibung viele Überschneidungen zur Geografie und zur Geschichte aufwies, gehörte mit zu denjenigen Wissenschaften, die den Nationalstaat erfanden. Sie konzipierten ihn als Container, sprich als abgeschlossenes Objekt, das durch die «Einheit von Staatsgewalt, Staatsvolk und Staatsgebiet» definiert wurde (\cite{woolf_statistics_1989}).  

Der Schule kam in diesem Prozess eine wichtige Rolle der Popularisierung zu. Nicht nur vermittelte sie den Schüler*innen durch die neuen Fächer wie Geschichte und Landeskunde, dass sie Teil eines Staatsvolks waren, die allgemeine Schulpflicht betonte auch die rechtliche Gleichstellung der Staatsbürger (\cite{criblez_erziehung_1998, dahn_learning_2015}). Die Verbindung zwischen Schule und Statistik ist noch enger. So wurde die Schulbildung selbst zum Objekt der statistischen Forschung. Der Zustand der Schulen und mehr noch die Frage nach den Verhältnissen der Schüler*innen, die Regelmässigkeit des Schulbesuchs, ihre Gesundheit, ihr Schulerfolg, werden vermessen und auf eindeutige Zahlen gebracht. Diese werden als Chiffren für dahinterstehende gesellschaftliche Zustände gelesen, welche staatliche Interventionen legitimieren, sei es zur Volksgesundheit, zur Verbesserung der Lehrerausbildung oder zur Bekämpfung von Armut und Alkoholismus in den Elternhäusern. Für die Schweiz ist dies bisher für die Zeit am Ende des 19.~Jahrhunderts dokumentiert (\cite{ruoss_zahlen_2018}), für einzelne Kantone auch früher. Diese Arbeit möchte versuchen, Hinweise auf die Erfindung der Schweizer Gesellschaft anhand der Schulstatistiken bereits für einen früheren Zeitraum zu finden.

\section{Zwischenfazit}

Vor dem Hintergrund des historischen Forschungsstands verspricht diese Arbeit in zwei Punkten einen neuen Beitrag: Zum einen ist die Geschichte der Schweizer Statistik im Zeitraum vor Gründung des Bundesstaats 1848 noch wenig erforscht. Anhand des Diskurses um die Volksschule kann untersucht werden, welchen Einfluss Statistik auf die Diskussion gesellschaftlicher Probleme hatte. Zum anderen ist die hier angewendete Methodik einer linguistisch informierten Korpusanalyse für die Geschichtswissenschaften ungewöhnlich. Im folgenden Kapitel wird deshalb die Analysemethode theoretisch begründet. 