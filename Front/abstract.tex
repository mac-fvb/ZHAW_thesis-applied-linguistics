% !TEX root = ../main.tex

%----------------------------------------------------------------------------------------
% ABSTRACT PAGE
%----------------------------------------------------------------------------------------
\begin{abstract}

\addchaptertocentry{\abstractname} % Add the abstract to the table of contents
\selectlanguage{english}
The thesis examines the possibilities of writing social history using the methods of corpus linguistics. It will do so with a case study of the Swiss-German educational discourse from around 1800 to the 1870s.

The main hypothesis is that the digitalisation of Switzerland began in the 19th century. After the French Revolution, the idea gained hold that people lived as equal citizens in a nation state. This population could only be described, recorded and controlled by ‘digital’ methods. ‘Rich’ social phenomena were reduced to numbers that were discrete and freely recombinable. These numbers could be combined in new ways to reveal patterns in social behaviour which had been invisible before. The method of generating these descriptions was statistics, which came to be the dominant mode how societies observed themselves. The expansion of elementary education was an important driver of this discovery of latent patterns of social behaviour. 

Unlike hermeneutic procedures of text and discourse analysis, this study attempts to pursue the research question from an inductive perspective. It is therefore also a contribution to the methodological discussion of how applied linguistics can provide insights into social history.

\hspace{2cm}

\selectlanguage{ngerman}
Die Arbeit untersucht die Möglichkeiten einer korpusgestützen Sozialgeschichte unter Verwendung korpuslinguistischer Methoden. Grundlegend ist die Hypothese, dass die Digitalisierung der Schweiz bereits im frühen 19. Jahrhundert einsetzt. Diese These wird anhand einer Untersuchung des deutsch\-schwei\-zerischen Bildungsdiskurses von etwa 1800 bis in die 1870er-Jahre verfolgt.

Das Ende der Ständestaaten erzeugte die Vorstellung von territorial definierten Staaten, in denen gleichberechtigte Bürger  zusammenlebten. Diese Bevölkerung war nur mit «digitalen» Methoden beschreib-, erfass- und steuerbar: Diskreten, frei re-kombinierbaren Signifikatoren, die «reiche» soziale Phänomene auf Zahlen reduzierten. Diese konnten auf vielfältige Weise in Bezug zueinander gesetzt werden. Die Methode, diese Beschreibungen zu erzeugen, war die Statistik, die sich in diesem Zeitraum als Modus der Selbstbeobachtung moderner Nationen etablierte. Sie erlaubte es, Gesellschaft als Abstraktum zu denken und Muster in scheinbar ungeregelten sozialen Phänomenen zu entdecken. 

Anders als hermeneutischen Verfahren der Text- und Diskursanalyse wird versucht, die Fragestellung aus einer induktiven Perspektive heraus zu verfolgen. Sie ist daher, neben der inhaltlichen Untersuchung, ein Beitrag zur Methodendiskussion, wie die Angewandte Linguistik Erkenntnisse zur Sozialgeschichte liefern kann.

\end{abstract}


%----------------------------------------------------------------------------------------
% German ABSTRACT PAGE
%----------------------------------------------------------------------------------------
%\begin{extraAbstract}
%\addchaptertocentry{\extraabstractname} % Add the abstract to the table of contents
%
%Die Zusammenfassung entspricht einer Miniaturversion des gesamten Dokuments. Gliedere sie ähnlich: Beginne mit dem Kontext und der %Motivation für das Projekt, einer kurzen Beschreibung der Methode und der verfügbaren Daten, Ihren Ergebnissen und den %Schlussfolgerungen. Beschränke dich auf eine Seite!    
%\end{extraAbstract}
